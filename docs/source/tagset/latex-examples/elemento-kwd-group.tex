%%% Execute with `pdflatex <file.tex>

%%%%% Preamble
\documentclass[10pt,a4paper]{article}

\usepackage{latex-basic}
\usepackage{scielostyle}

\definecolor{out}{rgb}{1,0.2,0.6}

\begin{document}

\printtitle 
\printauthor

      \renewcommand{\kwdgroupbeforeskip}{\medskip}
      \renewcommand{\kwdstyle}{\itshape\color{out}}
      \renewcommand{\kwdtitle}{Keywords:}
      \renewcommand{\kwdtitlestyle}{\noindent\bfseries\color{out}}
      \renewcommand{\kwd}[1]{{\kwdtitlestyle
                              \kwdtitle} 
                              \kwdstyle #1}
      \renewcommand{\kwdgroupbeforeskip}{\medskip}
      \renewenvironment{kwdgroup}{\kwdgroupbeforeskip}{}

%%%%%%%%%%%%%%%%%%%%%%%%%%%%%%%%%%%%%%%%%%%%%%%%%%%%%%%%%%%%%%%%%%%%%%%%%%%%%%%%%%%%%%%%%%%%%%
%%%%%%%%%%%%%%%%%%%%%%%%%%%%%%%%%%%%%%%%%%%%%%%%%%%%%%%%%%%%%%%%%%%%%%%%%%%%%%%%%%%%%%%%%%%%%%
            \renewcommand{\abstractname}{Nome do abstract}
            
            \begin{abstract}   
                  Verificar a sensibilidade e especificidade das curvas de fluxo-volume na
                  detecção de obstrução da via aérea central (OVAC), e se os critérios
                  qualitativos e quantitativos da curva se relacionam com a localização, o
                  tipo e o grau de obstrução. Métodos: Durante quatro meses foram
                  selecionados, consecutivamente, indivíduos com indicação para
                  broncoscopia. Todos efetuaram avaliação clínica, preenchimento de escala
                  de dispneia, curva de fluxo-volume e broncoscopia num intervalo de uma
                  semana. Quatro revisores classificaram a morfologia da curva sem
                  conhecimento dos dados quantitativos, clínicos e broncoscopicos. Um
                  quinto revisor averiguou os critérios morfológicos e quantitativos.

                  \begin{kwdgroup}
                  \kwd{Chagas disease, Quality of life, Health-related quality of 
                  life, Cardiomyopathy, Determining factors}
                  \end{kwdgroup}
            \end{abstract}


%%%%%%%%%%%%%%%%%%%%%%%%%%%%%%%%%%%%%%%%%%%%%%%%%%%%%%%%%%%%%%%%%%%%%%%%%%%%%%%%%%%%%%%%%%%%%%
%%%%%%%%%%%%%%%%%%%%%%%%%%%%%%%%%%%%%%%%%%%%%%%%%%%%%%%%%%%%%%%%%%%%%%%%%%%%%%%%%%%%%%%%%%%%%%

\lipsum
%\input{latex-example-lipsum}
\end{document}
